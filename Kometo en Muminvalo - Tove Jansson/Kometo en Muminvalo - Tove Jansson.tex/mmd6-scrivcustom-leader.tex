% Optimally you should use XeLaTeX to typeset it, using system fonts. If you use pdflatex, standard LaTeX sans fonts will be used instead.
%
% This makes use of some stock MultiMarkdown LaTeX boilerplate. Thanks go to Fletcher Penney for creating a useful and simple system to extend from.

\documentclass[oneside,article,14pt,dvipsnames]{memoir}

\usepackage{layouts}[2001/04/29]
\usepackage[svgnames]{xcolor}
\definecolor{coolgrey}{Hsb}{293, 0.0, 0.5}
\definecolor{plumbgrey}{Hsb}{314, 0.3, 0.5}
\definecolor{redgrey}{Hsb}{337, 0.31, 0.6}
\usepackage{verse}
\usepackage{wrapfig}
\usepackage{sectionbreak}
\usepackage{perpage} %the perpage package
\MakePerPage{footnote} %the perpage package command
\usepackage{xpatch}
\usepackage{fancyvrb}
\usepackage{graphicx}
\usepackage{booktabs}
\usepackage{tabulary}
%\usepackage{listings}
\usepackage[sort&compress]{natbib}
\usepackage[normalem]{ulem}
\usepackage{adjustbox}
\usepackage[esperanto]{babel}
\usepackage{amssymb}
\usepackage[acronym]{glossaries}
\usepackage[utf8]{inputenc}
\usepackage[labelformat=empty]{caption}
\usepackage{calligra}
\usepackage{tikz}
\usetikzlibrary{matrix,fit,chains,calc,scopes}            
\usepackage{tcolorbox}
\tcbuselibrary{skins}
\usepackage{auto-pst-pdf} %To compile psvectorian directly
\usepackage{psvectorian}
\usepackage[all]{nowidow}
\glstoctrue
\makeglossaries
\makeindex

\usepackage{pdfpages}
\usepackage[T1]{fontenc}
\usepackage{fontenc,unicode-math}
%\usepackage{fontspec}
\setmainfont[Ligatures=TeX]{TeX Gyre Schola}
%\usepackage{beton}
%\renewcommand{\bfdefault}{sbc}
\usepackage[scale=0.89]{tgheros} % Helvetica is too big


%	\renewcommand{\familydefault}{\sfdefault}
	\newcommand{\helvetican}{}
	\newcommand{\helveticanl}{}

% Body Text Formatting
\linespread{1.2}
\setlength{\abnormalparskip}{0em}
\setlength{\parindent}{1.5em}


% Footnotes
\setlength{\footmarkwidth}{1.8em}
\setlength{\footmarksep}{0em}
\footmarkstyle{\footnotesize{#1}.\hfill}
\setfootins{16pt}{16pt}
\setlength{\footnotesep}{16pt}


%
%	8.5 x 11 layout for memoir-based documents
%   As defined in the stock MultiMarkdown system
%
%%% need more space for ToC page numbers
\setpnumwidth{2.55em}
\setrmarg{3.55em}

%%% need more space for ToC section numbers
\cftsetindents{part}{0em}{3em}
\cftsetindents{chapter}{0em}{3em}
\cftsetindents{section}{3em}{3em}
\cftsetindents{subsection}{4.5em}{3.9em}
\cftsetindents{subsubsection}{8.4em}{4.8em}
\cftsetindents{paragraph}{10.7em}{5.7em}
\cftsetindents{subparagraph}{12.7em}{6.7em}

%%% need more space for LoF numbers
\cftsetindents{figure}{0em}{3.0em}

%%% and do the same for the LoT
\cftsetindents{table}{0em}{3.0em}

%%% set up the page layout
\settrimmedsize{\stockheight}{\stockwidth}{*}	% Use entire page
\settrims{0pt}{0pt}

% Comment out the following command and replace it with the second, below it, if you intend to use CriticMarkup's margin notes, or marginalia of any sort.
\setlrmarginsandblock{1in}{1in}{*}
% \setlrmarginsandblock{1in}{2.5in}{*}
\setulmarginsandblock{1in}{1in}{*}

\setmarginnotes{17pt}{1.5in}{\onelineskip}
\setheadfoot{\onelineskip}{2\onelineskip}
\setheaderspaces{*}{2\onelineskip}{*}
\checkandfixthelayout

\VerbatimFootnotes

% Section Headings
\setsecheadstyle{\helveticanl\LARGE\raggedright\textcolor{ForestGreen}}
\setsubsecheadstyle{\helveticanl\large\raggedright\textcolor{ForestGreen}}
\setsubsubsecheadstyle{\helveticanl\normalsize\raggedright\textcolor{ForestGreen}}

\maxsecnumdepth{chapter}
\setsecnumdepth{chapter}
\settocdepth{section}

% Spacing Model
% Use "negative" values for the beforeXskip settings; this indicates the following paragraph should have its indent suppressed.
\setbeforesecskip{-14pt}
\setaftersecskip{12pt}
\setbeforesubsecskip{-14pt}
\setaftersubsecskip{12pt}
\setbeforesubsubsecskip{-14pt}
\setaftersubsubsecskip{6pt}

% Chapter heading style
\makechapterstyle{modern-style}{
    \renewcommand*{\chaptitlefont}{\Huge\centering\normalfont}
    \renewcommand*{\chapnumfont}{\Huge\raggedright\mdseries}
	\renewcommand*{\chapnamefont}{\chapnumfont}
	\renewcommand*{\printchaptername}{\chapnamefont\color{ForestGreen}}
	\renewcommand*{\printchapternonum}{\chapnamefont\color{ForestGreen}}
	\renewcommand*{\printchapternum}{}
	\renewcommand*{\chapterheadstart}{\newpage \vspace*{4em}}

}
\chapterstyle{modern-style}

%\renewcommand*{\chapnumfont}{\Huge\raggedright\centering\color{ForestGreen}}
%\renewcommand*{\chapnamefont}{\chapnumfont}

% Part Breaks


\renewcommand{\partnamefont}{}
\renewcommand{\printpartname}{}
\renewcommand{\printpartnum}{}
\renewcommand{\partnumfont}{\partnamefont}
\renewcommand{\beforepartskip}{\null\vfil\thispagestyle{empty}}
\renewcommand{\midpartskip}{\par\vspace{6pt}}
% Comment the following line if you wish to print the name of the part
\renewcommand{\printparttitle}{\chapnamefont\centering\color{ForestGreen}}
\renewcommand{\afterpartskip}{\par\vspace{0pt}}


\renewcommand{\cftpartleader}{\cftdotfill{\cftdotsep}}
\renewcommand{\cftchapterleader}{\cftdotfill{\cftdotsep}}

\pagestyle{plain}
               
\renewcommand{\sectionbreak}{\fancybreak{\color{ForestGreen}\textbf{\star}}}